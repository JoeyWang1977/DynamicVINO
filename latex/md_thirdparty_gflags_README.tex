\href{https://travis-ci.org/gflags/gflags}{\tt } \href{https://ci.appveyor.com/project/schuhschuh/gflags/branch/master}{\tt }

The documentation of the gflags library is available online at \href{https://gflags.github.io/gflags/}{\tt https\+://gflags.\+github.\+io/gflags/}.

\subsection*{11 July 2017 }

I\textquotesingle{}ve just released gflags 2.\+2.\+1.

This maintenance release primarily fixes build issues on Windows and false alarms reported by static code analyzers.

Please report any further issues with this release using the Git\+Hub issue tracker.

\subsection*{25 November 2016 }

I\textquotesingle{}ve finally released gflags 2.\+2.\+0.

This release adds support for use of the gflags library as external dependency not only in projects using C\+Make, but also \href{https://bazel.build/}{\tt Bazel}, or \href{https://www.freedesktop.org/wiki/Software/pkg-config/}{\tt pkg-\/config}. One new minor feature is added in this release\+: when a command flag argument contains dashes, these are implicitly converted to underscores. This is to allow those used to separate words of the flag name by dashes to do so, while the flag variable names are required to use underscores.

Memory leaks reported by valgrind should be resolved by this release. This release fixes build errors with MS Visual Studio 2015.

Please report any further issues with this release using the Git\+Hub issue tracker.

\subsection*{24 March 2015 }

I\textquotesingle{}ve just released gflags 2.\+1.\+2.

This release completes the namespace change fixes. In particular, it restores binary A\+BI compatibility with release version 2.\+0. The deprecated \char`\"{}google\char`\"{} namespace is by default still kept as primary namespace while symbols are imported into the new \char`\"{}gflags\char`\"{} namespace. This can be overridden using the C\+Make variable \hyperlink{namespaceGFLAGS__NAMESPACE}{G\+F\+L\+A\+G\+S\+\_\+\+N\+A\+M\+E\+S\+P\+A\+CE}.

Other fixes of the build configuration are related to the (patched) C\+Make modules Find\+Threads.\+cmake and Check\+Type\+Size.\+cmake. These have been removed and instead the C language is enabled again even though gflags is written in C++ only.

This release also marks the complete move of the gflags project from Google Code to Git\+Hub. Email addresses of original issue reporters got lost in the process. Given the age of most issue reports, this should be negligable.

Please report any further issues using the Git\+Hub issue tracker.

\subsection*{30 March 2014 }

I\textquotesingle{}ve just released gflags 2.\+1.\+1.

This release fixes a few bugs in the configuration of \hyperlink{gflags__declare_8h}{gflags\+\_\+declare.\+h} and adds a separate G\+F\+L\+A\+G\+S\+\_\+\+I\+N\+C\+L\+U\+D\+E\+\_\+\+D\+IR C\+Make variable to the build configuration. Setting \hyperlink{namespaceGFLAGS__NAMESPACE}{G\+F\+L\+A\+G\+S\+\_\+\+N\+A\+M\+E\+S\+P\+A\+CE} to \char`\"{}google\char`\"{} no longer changes also the include path of the public header files. This allows the use of the library with other Google projects such as glog which still use the deprecated \char`\"{}google\char`\"{} namespace for the gflags library, but include it as \char`\"{}gflags/gflags.\+h\char`\"{}.

\subsection*{20 March 2014 }

I\textquotesingle{}ve just released gflags 2.\+1.

The major changes are the use of C\+Make for the build configuration instead of the autotools and packaging support through C\+Pack. The default namespace of all C++ symbols is now \char`\"{}gflags\char`\"{} instead of \char`\"{}google\char`\"{}. This can be configured via the \hyperlink{namespaceGFLAGS__NAMESPACE}{G\+F\+L\+A\+G\+S\+\_\+\+N\+A\+M\+E\+S\+P\+A\+CE} variable.

This release compiles with all major compilers without warnings and passed the unit tests on Ubuntu 12.\+04, Windows 7 (Visual Studio 2008 and 2010, Cygwin, Min\+GW), and Mac OS X (Xcode 5.\+1).

The S\+VN repository on Google Code is now frozen and replaced by a Git repository such that it can be used as Git submodule by projects. The main hosting of this project remains at Google Code. Thanks to the distributed character of Git, I can push (and pull) changes from both Git\+Hub and Google Code in order to keep the two public repositories in sync. When fixing an issue for a pull request through either of these hosting platforms, please reference the issue number as \href{https://code.google.com/p/support/wiki/IssueTracker#Integration_with_version_control}{\tt described here}. For the further development, I am following the \href{http://nvie.com/posts/a-successful-git-branching-model/}{\tt Git branching model} with feature branch names prefixed by \char`\"{}feature/\char`\"{} and bugfix branch names prefixed by \char`\"{}bugfix/\char`\"{}, respectively.

Binary and source \href{https://github.com/schuhschuh/gflags/releases}{\tt packages} are available on Git\+Hub.

\subsection*{14 January 2014 }

The migration of the build system to C\+Make is almost complete. What remains to be done is rewriting the tests in Python such they can be executed on non-\/\+Unix platforms and splitting them up into separate C\+Test tests. Though merging these changes into the master branch yet remains to be done, it is recommended to already start using the \href{https://github.com/schuhschuh/gflags/tree/cmake-migration}{\tt cmake-\/migration} branch.

\subsection*{20 April 2013 }

More than a year has past since I (Andreas) took over the maintenance for {\ttfamily gflags}. Only few minor changes have been made since then, much to my regret. To get more involved and stimulate participation in the further development of the library, I moved the project source code today to \href{https://github.com/schuhschuh/gflags}{\tt Git\+Hub}. I believe that the strengths of \href{http://git-scm.com/}{\tt Git} will allow for better community collaboration as well as ease the integration of changes made by others. I encourage everyone who would like to contribute to send me pull requests. Git\textquotesingle{}s lightweight feature branches will also provide the right tool for more radical changes which should only be merged back into the master branch after these are complete and implement the desired behavior.

The S\+VN repository remains accessible at Google Code and I will keep the master branch of the Git repository hosted at Git\+Hub and the trunk of the Subversion repository synchronized. Initially, I was going to simply switch the Google Code project to Git, but in this case the S\+VN repository would be frozen and force everyone who would like the latest development changes to use Git as well. Therefore I decided to host the public Git repository at Git\+Hub instead.

Please continue to report any issues with gflags on Google Code. The Git\+Hub project will only be used to host the Git repository.

One major change of the project structure I have in mind for the next weeks is the migration from autotools to \href{http://www.cmake.org/}{\tt C\+Make}. Check out the (unstable!) \href{https://github.com/schuhschuh/gflags/tree/cmake-migration}{\tt cmake-\/migration} branch on Git\+Hub for details.

\subsection*{25 January 2012 }

I\textquotesingle{}ve just released gflags 2.\+0.

The {\ttfamily google-\/gflags} project has been renamed to {\ttfamily gflags}. I (csilvers) am stepping down as maintainer, to be replaced by Andreas Schuh. Welcome to the team, Andreas! I\textquotesingle{}ve seen the energy you have around gflags and the ideas you have for the project going forward, and look forward to having you on the team.

I bumped the major version number up to 2 to reflect the new community ownership of the project. All the \href{ChangeLog.txt}{\tt changes} are related to the renaming. There are no functional changes from gflags 1.\+7. In particular, I\textquotesingle{}ve kept the code in the namespace {\ttfamily google}, though in a future version it should be renamed to {\ttfamily gflags}. I\textquotesingle{}ve also kept the {\ttfamily /usr/local/include/google/} subdirectory as synonym of {\ttfamily /usr/local/include/gflags/}, though the former name has been obsolete for some time now.

\subsection*{18 January 2011 }

The {\ttfamily google-\/gflags} Google Code page has been renamed to {\ttfamily gflags}, in preparation for the project being renamed to {\ttfamily gflags}. In the coming weeks, I\textquotesingle{}ll be stepping down as maintainer for the gflags project, and as part of that Google is relinquishing ownership of the project; it will now be entirely community run. The name change reflects that shift.

\subsection*{20 December 2011 }

I\textquotesingle{}ve just released gflags 1.\+7. This is a minor release; the major change is that {\ttfamily Command\+Line\+Flag\+Info} now exports the address in memory where the flag is located. There has also been a bugfix involving very long --help strings, and some other minor \href{ChangeLog.txt}{\tt changes}.

\subsection*{29 July 2011 }

I\textquotesingle{}ve just released gflags 1.\+6. The major new feature in this release is support for setting version info, so that --version does something useful.

One minor change has required bumping the library number\+: {\ttfamily Reparse\+Commandline\+Flags} now returns {\ttfamily void} instead of {\ttfamily int} (the int return value was always meaningless). Though I doubt anyone ever used this (meaningless) return value, technically it\textquotesingle{}s a change to the A\+BI that requires a version bump. A bit sad.

There\textquotesingle{}s also a procedural change with this release\+: I\textquotesingle{}ve changed the internal tools used to integrate Google-\/supplied patches for gflags into the opensource release. These new tools should result in more frequent updates with better change descriptions. They will also result in future {\ttfamily Change\+Log} entries being much more verbose (for better or for worse).

See the \href{ChangeLog.txt}{\tt Change\+Log} for a full list of changes for this release.

\subsection*{24 January 2011 }

I\textquotesingle{}ve just released gflags 1.\+5. This release has only minor changes from 1.\+4, including some slightly better reporting in --help, and an new memory-\/cleanup function that can help when running gflags-\/using libraries under valgrind. The major change is to fix up the macros ({\ttfamily D\+E\+F\+I\+N\+E\+\_\+bool} and the like) to work more reliably inside namespaces.

If you have not had a problem with these macros, and don\textquotesingle{}t need any of the other changes described, there is no need to upgrade. See the \href{ChangeLog.txt}{\tt Change\+Log} for a full list of changes for this release.

\subsection*{11 October 2010 }

I\textquotesingle{}ve just released gflags 1.\+4. This release has only minor changes from 1.\+3, including some documentation tweaks and some work to make the library smaller. If 1.\+3 is working well for you, there\textquotesingle{}s no particular reason to upgrade.

\subsection*{4 January 2010 }

I\textquotesingle{}ve just released gflags 1.\+3. gflags now compiles under M\+S\+VC, and all tests pass. I {\bfseries really} never thought non-\/unix-\/y Windows folks would want gflags, but at least some of them do.

The major news, though, is that I\textquotesingle{}ve separated out the python package into its own library, \href{http://code.google.com/p/python-gflags}{\tt python-\/gflags}. If you\textquotesingle{}re interested in the Python version of gflags, that\textquotesingle{}s the place to get it now.

\subsection*{10 September 2009 }

I\textquotesingle{}ve just released gflags 1.\+2. The major change from gflags 1.\+1 is it now compiles under Min\+GW (as well as cygwin), and all tests pass. I never thought Windows folks would want unix-\/style command-\/line flags, since they\textquotesingle{}re so different from the Windows style, but I guess I was wrong!

The other changes are minor, such as support for --htmlxml in the python version of gflags.

\subsection*{15 April 2009 }

I\textquotesingle{}ve just released gflags 1.\+1. It has only minor changes fdrom gflags 1.\+0 (see the \href{ChangeLog.txt}{\tt Change\+Log} for details). The major change is that I moved to a new system for creating .deb and .rpm files. This allows me to create x86\+\_\+64 deb and rpm files.

In the process of moving to this new system, I noticed an inconsistency\+: the tar.\+gz and .rpm files created libraries named libgflags.\+so, but the deb file created libgoogle-\/gflags.\+so. I have fixed the deb file to create libraries like the others. I\textquotesingle{}m no expert in debian packaging, but I believe this has caused the package name to change as well. Please let me know (at \mbox{[}\href{mailto:google-gflags@googlegroups.com}{\tt google-\/gflags@googlegroups.\+com} \href{mailto:google-gflags@googlegroups.com}{\tt google-\/gflags@googlegroups.\+com}\mbox{]}) if this causes problems for you -- especially if you know of a fix! I would be happy to change the deb packages to add symlinks from the old library name to the new (libgoogle-\/gflags.\+so -\/$>$ libgflags.\+so), but that is beyond my knowledge of how to make .debs.

If you\textquotesingle{}ve tried to install a .rpm or .deb and it doesn\textquotesingle{}t work for you, let me know. I\textquotesingle{}m excited to finally have 64-\/bit package files, but there may still be some wrinkles in the new system to iron out.

\subsection*{1 October 2008 }

gflags 1.\+0rc2 was out for a few weeks without any issues, so gflags 1.\+0 is now released. This is much like gflags 0.\+9. The major change is that the .h files have been moved from {\ttfamily /usr/include/google} to {\ttfamily /usr/include/gflags}. While I have backwards-\/compatibility forwarding headeds in place, please rewrite existing code to say 
\begin{DoxyCode}
1 #include <gflags/gflags.h>
\end{DoxyCode}
 instead of 
\begin{DoxyCode}
1 #include <google/gflags.h>
\end{DoxyCode}


I\textquotesingle{}ve kept the default namespace to google. You can still change with with the appropriate flag to the configure script ({\ttfamily ./configure -\/-\/help} to see the flags). If you have feedback as to whether the default namespace should change to gflags, which would be a non-\/backwards-\/compatible change, send mail to {\ttfamily google-\/gflags@googlegroups.\+com}!

Version 1.\+0 also has some neat new features, like support for bash commandline-\/completion of help flags. See the \href{ChangeLog.txt}{\tt Change\+Log} for more details.

If I don\textquotesingle{}t hear any bad news for a few weeks, I\textquotesingle{}ll release 1.\+0-\/final. 